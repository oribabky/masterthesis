\chapter{Literature Review}
%Write about the theory used in the research.
\emph{The chapter gives both general and specific information on theory used for this project.  Mathematical statistics, regression and machine learning are covered in the first three sections, providing a general understanding of the field of study. Specific machine learning models are explained in the final three sections of the chapter. }


\section{Machine learning}
	Programming is typically about writing sequences of machine instructions, on a certain level of abstraction, for some software application. Once the application is used, it will determine logically what instructions to send to the machine, depending on what input is given from the user and what state the application is in etc. But as software applications grow in terms of lines of code, the amount of bugs and complexity increases, which in turn results in reduced software performance and readability \cite{IP:1}. 

Although today's programming languages typically offer some level of abstraction, and thus reduced complexity, it is still generally required of the programmer to account for every possible logical outcome of the software which has unwanted behavior. Today there are algorithms that improve automatically from past experiences, much like us humans do.

%Complexity can be mitigated by programming in programming languages. Instead of writing binary machine instructions, it's possible to write logical statements, expressions etc. in a programming language like C and compile the code back to machine instructions upon execution. 
	
Machine learning is formally defined by Mitchell \cite{BOOK:2}: 
	"A computer program is said to learn from experience $E$ with respect to some class of tasks $T$ and performance measure $P$ if its performance at tasks in $T$, as measured by $P$, improves with experience $E$".
	What this means is that machine learning algorithms is a way of 
	
	Two of the most common machine learning tasks: 
	\begin{itemize}
		\item{Classification:} The computer is asked to specify which category a certain input belongs to. An example of a classification task is  
		\item{Regression:} asdfsdf
	\end{itemize} \cite{BOOK:1}
	%why should machine learning be used for this purpose? \cite{BOOK:1}
	%how do we apply machine learning to solve the issue?
	Something that \cite{WEBSITE:1}
\subsection{Neural networks}

