% !TEX TS-program = pdflatex
% !TEX encoding = UTF-8 Unicode

% This is a simple template for a LaTeX document using the "article" class.
% See "book", "report", "letter" for other types of document.

\documentclass[oneside, 11pt]{report} % use larger type; default would be 10pt

\usepackage[utf8]{inputenc} % set input encoding (not needed with XeLaTeX)

%%% Examples of Article customizations
% These packages are optional, depending whether you want the features they provide.
% See the LaTeX Companion or other references for full information.

%%% PAGE DIMENSIONS
\usepackage{geometry} % to change the page dimensions
\geometry{a4paper} % or letterpaper (US) or a5paper or....
% \geometry{margin=2in} % for example, change the margins to 2 inches all round
% \geometry{landscape} % set up the page for landscape
%   read geometry.pdf for detailed page layout information

\usepackage{graphicx} % support the \includegraphics command and options

% \usepackage[parfill]{parskip} % Activate to begin paragraphs with an empty line rather than an indent

%%% PACKAGES
\usepackage[english,swedish]{babel}
\usepackage{booktabs} % for much better looking tables
\usepackage{array} % for better arrays (eg matrices) in maths
\usepackage{paralist} % very flexible & customisable lists (eg. enumerate/itemize, etc.)
\usepackage{verbatim} % adds environment for commenting out blocks of text & for better verbatim
\usepackage{subfig} % make it possible to include more than one captioned figure/table in a single float
% These packages are all incorporated in the memoir class to one degree or another...

%%% CUSTOM PACKAGES
\usepackage{fixltx2e}
\usepackage{float}

%%% HEADERS & FOOTERS
\usepackage{fancyhdr} % This should be set AFTER setting up the page geometry
\pagestyle{fancy} % options: empty , plain , fancy
\renewcommand{\headrulewidth}{0pt} % customise the layout...
\lhead{}\chead{}\rhead{}
\lfoot{}\cfoot{\thepage}\rfoot{}

%%% SECTION TITLE APPEARANCE
\usepackage{sectsty}
\allsectionsfont{\sffamily\mdseries\upshape} % (See the fntguide.pdf for font help)
% (This matches ConTeXt defaults)

%%% ToC (table of contents) APPEARANCE
\usepackage[nottoc,notlof,notlot]{tocbibind} % Put the bibliography in the ToC
\usepackage[titles,subfigure]{tocloft} % Alter the style of the Table of Contents
\renewcommand{\cftsecfont}{\rmfamily\mdseries\upshape}
\renewcommand{\cftsecpagefont}{\rmfamily\mdseries\upshape} % No bold!

%Set graphics path
\graphicspath{ {img/} }

%add references
%\addbibresource{references.bib}
% setup biblatex/bibliography
%\usepackage[sorting=none]{biblatex}
\usepackage[
	backend=bibtex,
	style=ieee,
	sorting=none,
	dashed=false
	]{biblatex}
\bibliography{references}

%packages for tables and figures
\usepackage{makecell}
\usepackage[margin=10pt,font=small,labelfont=bf,
labelsep=endash]{caption}

%%% subsubsubsection settings
\usepackage{titlesec}

% allow degree symbol
\usepackage{gensymb}

\setcounter{secnumdepth}{4}

\titleformat{\paragraph}
{\normalfont\normalsize\bfseries}{\theparagraph}{1em}{}
\titlespacing*{\paragraph}
{0pt}{3.25ex plus 1ex minus .2ex}{1.5ex plus .2ex}

\usepackage{amsmath}

\newenvironment{poliabstract}[1]
   {\renewcommand{\abstractname}{#1}\begin{abstract}}
   {\end{abstract}}

%%% END Article customizations

%%% The "real" document content comes below...

\title{Using supervised learning algorithms to model the behavior of Road Weather Information System sensors}
\author{Tobias Axelsson}
%\date{} % Activate to display a given date or no date (if empty),
         % otherwise the current date is printed 

\begin{document}
%\begin{titlepage}
\pagenumbering{gobble}
%\maketitle

\chapter*{Acknowledgements}
	%\thispagestyle{empty}
	This thesis marks the final step for me to achieve a master's degree in computer science. Six years have soon passed since I started studying at Luleå university of Technology, a time in my life where I have learned a lot and met wonderful people. I would like to thank classmates, friends and teachers which I have got to know during my years in Luleå for making this a memorable and educative journey. 

	This thesis would not be possible without guidance from my supervisors. Firstly I would like to thank Niklas Karvonen at LTU for the work of reviewing my thesis, providing advice on how to write academically, and for all the machine learning teachings which proved valuable for a machine learning newcomer like myself. Secondly I would like to thank Johan Casselgren at LTU for giving me the opportunity of carrying out this thesis with Trafikverket, and for all the help I have received throughout the process. I would also like to thank Jonas Hallenberg at Trafikverket for providing a dataset which I could work with, and for answering any questions I had regarding Road Weather Information Systems throughout this project.\\\\
	Luleå, June 2018\\\\
	Tobias Axelsson

\selectlanguage{english}	
\begin{poliabstract}{Abstract} 
	Trafikverket, the agency in charge of state road maintenance in Sweden, have a number of so-called Road Weather Information Systems (RWIS). The main purpose of the stations is to provide winter road maintenance workers with information to decide when roads need to be plowed and/or salted. Each RWIS have a number of sensors which make road weather-related measurements every 30 minutes. One of the sensors is dug into the road which can cause traffic disturbances and be costly for Trafikverket. Other RWIS sensors fail occasionally.

	This project aims at modelling a set of RWIS sensors using supervised machine learning algorithms. The sensors that are of interest to model are: Optic Eye, Track Ice Road Sensor (TIRS) and DST111. Optic Eye measures precipitation type and precipitation amount. Both TIRS and DST111 measure road surface temperature. The difference between TIRS and DST111 is that the former is dug into the road, and DST111 measures road surface temperature from a distance via infrared laser. Any supervised learning algorithm trained to model a given measurement made by a sensor, may only train on measurements made by the other sensors as input features. Measurements made by TIRS may not be used as input in modelling other sensors, since it is desired to see if TIRS can be removed. The following input features may also be used for training: road friction, road surface condition and timestamp.

	Scikit-learn was used as machine learning software in this project. An experimental approach was chosen to achieve the project results: A pre-determined set of supervised algorithms were compared using different amount of top relevant input features and different hyperparameter settings. Prior to achieving the results, a data preparation process was conducted. Observations with suspected or definitive errors were removed in this process. During the data preparation process, the timestamp feature was transformed into two new features: month and hour.

	The results in this project show that precipitation type was best modelled using Classification And Regression Tree (CART) on Scikit-learn default settings, achieving a performance score of $\overline{F1}_{test} = 0.46$ and accuracy $=0.84$ using road surface condition, road friction, DST111 road surface temperature, hour and month as input features. Precipitation amount was best modelled using $k$-Nearest Neighbor (kNN); with $k=64$ and road friction used as the only input feature, a performance score of $MSE_{test}=0.31$ was attained. TIRS road surface temperature was best modelled with Multi-Layer Perceptron (MLP) using 64 hidden nodes and DST111 road surface temperature, road surface condition, road friction, month, hour and precipitation type as input features, with which a performance score of $MSE_{test}=0.88$ was achieved. DST111 road surface temperature was best modelled using Random forest on Scikit-learn default settings with road surface condition, road friction, month, precipitation type and hour as input features, achieving a performance score of $MSE_{test}=10.16$.
\end{poliabstract}

\selectlanguage{swedish}
\begin{poliabstract}{Sammanfattning}
		Trafikverket är den organisation som ansvarar för underhåll av statliga vägar i Sverige. Till hjälp för att utföra detta har de ett antal så kallade Vägväderinformationssystem (VVIS) som framförallt syftar till att underlätta arbetet för de som arbetar med vintervägunderhåll. Informationen används för att bestämma om vägar behöver skottas och/eller saltas. Varje VVIS har ett antal sensorer som gör väder-relaterade mätningar var trettionde minut. En av dessa sensorer grävs ned i marken, vilket kan störa trafikflöden och införa kostnader för Trafikverket. Övriga sensorer fallerar relativt ofta, vilket kan ställa till problem för vintervägunderhållsarbetarna om de behöver information just då.

	Det här projektet syftar till att modellera ett antal VVIS sensorer med hjälp av övervakade maskininlärningsalgoritmer. Följande sensorer är intressanta att modellera: Optic Eye, Track Ice Road Sensor (TIRS) och DST111. Optic Eye mäter nederbördstyp och nederbördsmängd. Både TIRS och DST111 mäter väglagstemperatur, skillnaden är att TIRS är nedgrävd i vägbanan, medan DST111 mäter på avstånd via infraröd laser. En sensor som modelleras av en övervakad maskininlärningsalgoritm, får träna på data från övriga sensorer, förutom från TIRS då det är intressant att se om denna kan tas bort. Dessutom får följande input features användas för träning: vägfriktion, väglagstyp och tidsstämpel.

	Scikit-learn användes som maskininlärningsprogram i det här projektet. En experimentell strategi användes för att uppnå projektets resultat: Ett förbestämt antal algoritmer jämfördes där antalet mest relevanta input features och hyperparametrar varierades. En dataförberedelseprocess genomfördes innan resultaten nåddes. Observationer med antingen misstänkta eller definitiva fel togs bort från datasetet i detta steg. Under dataförberedelseprocessen transformerades även tidsstämpel som feature till två nya features: månad och timma.

	Resultaten i detta projektet visar att nederbördstyp modellerades bäst genom att använda Classification And Regression Tree (CART) på standardinställningar i Scikit-learn. En prestanda på $\overline{F1}_{test} = 0.46$ och noggrannhet $=0.84$ uppnåddes när väglagstyp, vägfriktion, DST111 väglagstemperatur, timma och månad användes som input features. Nederbördsmängd modellerades bäst genom att använda $k$-Nearest Neighbor (kNN) med $k=64$ och vägfriktion som input feature. Med detta uppnåddes en prestanda på $MSE_{test} = 0.31$. TIRS väglagstemperatur modellerades bäst genom att använda Multi-Layer Perceptron (MLP) med 64 gömda noder och DST111 väglagstemperatur, väglagstyp, vägfriktion, månad, timma och nederbördstyp som input features då en prestanda på $MSE_{test}=0.88$ åstadkoms. DST111 modellerades bäst genom att använda Random forest på standardinställningar i Scikit-learn och med väglagstyp, vägfriktion, månad, nederbördstyp och timma som input features. Med detta nåddes en prestanda på $MSE_{test}=10.16$.
\end{poliabstract}

\selectlanguage{english}


\tableofcontents
\newpage
%\end{titlepage}
\pagenumbering{arabic}

\chapter{Introduction}
\section{Background}
\begin{itemize}
	\item Why improve road condition monitoring?
	\item what is Trafikverket and what do they do? 
	\item weatherstations, what are they? What's the road surface temperature sensor?
	\item why is it desirable to simulate the road sensor?
	\item what is machine learning?
	\item why should machine learning be used for this purpose?
\end{itemize}

\section{Objective}
	The objective is to determine if a road surface temperature sensor can be simulated with prediction models based on data from road weather information systems.
\section{Scope}

\section{Thesis structure}


\chapter{Literature Review}
%Write about the theory used in the research.
\emph{The chapter gives both general and specific information on theory used for this project.  Mathematical statistics, regression and machine learning are covered in the first three sections, providing a general understanding of the field of study. Specific machine learning models are explained in the final three sections of the chapter. }


\section{Machine learning}
	Programming is typically about writing explicit sequences of machine instructions for some software application. Once the application is used, it will decide logically what instructions to send to the machine depending on what input is given from the user, its current state etc. But as software becomes complex the %HERE%
	Machine learning is formally defined by Mitchell \cite{BOOK:2}: 
	A computer program is said to learn from experience $E$ with respect to some class of tasks $T$ and performance measure $P$ if its performance at tasks in $T$, as measured by $P$, improves with experience $E$.
	What this means is that machine learning algorithms is a way of 
	
	Two of the most common machine learning tasks: 
	\begin{itemize}
		\item{Classification:} The computer is asked to specify which category a certain input belongs to. An example of a classification task is  
		\item{Regression:} asdfsdf
	\end{itemize} \cite{BOOK:1}
	%why should machine learning be used for this purpose? \cite{BOOK:1}
	%how do we apply machine learning to solve the issue?
	Something that \cite{WEBSITE:1}
\subsection{Neural networks}



\chapter{Method}
%describe methods. One with similar background should be able to carry out the research based on this chapter.
\emph{The chapter covers strategies and methods used to achieve the objective of the project. Reasons for each choice of method or strategy are motivated and described in the sections, which are ordered chronologically.}

%% what methods are used to carry out the research. Describe that either several machine learning models are used and compared or one is used and the work involved improving it etc.

% Under rubriken ”Metod” anges vilken metod/metoder som använts för att få fram resultaten som presenteras i texten. Du bör också analysera valet av metod genom att motivera varför metoden fungerar bäst i sammanhanget. Beskrivningen av metod/metoder ska vara så noggranna och utförliga att gången i arbetsprocessen ska kunna upprepas.

% skriva hur skall datan pre-processas?

% skriva hur skall dataset splittas? random? cross-validation?

\section{Data overview and test-implementation}
	%look at the data provided
	%behövs mer data?  	%We also demonstrate that HOG templates have a relatively small effective capacity; one can train accurate HOG templates with 100–200 positive examples (rather than thousands of examples as is typically done \cite{ARTICLE:18}
	% Varför inte data från hela sverige: det är intressant att studera dst111 och dsc111, dessa finns endast tillsammans på denna sträcka => geografiskt avgränsad data

	%obalanserad data
\section{Literature study}
	%describe learning process as well

\section{Choice of algorithms}
	%why choose the ones we did?

\section{Choice of model validation technique}
	%why split and not kfold?
		%overfitting is detected in this project by comparing training fit to test fit, this is done easily in holdout
		%k-fold has longer run-time
\section{Data preparation experiment setup} \label{sec:exp_setups}
	
	%standard classification experiment setup, seed = 7, strat-fold, algorithms
	%standard regression experiment setup: seed = 7, k-fold, algorithms
	%holdout classification experiment setup, seed = 7, holdout, 80% training, 20% test, algorithms

	%algorithms

	%standard features time, dst111,...
	%feature engineered features month, hour, ...
\section{Experimental methodology}
	\subsection{Data preparation}
	\begin{itemize}
		\item Remove all errors
		\item Study data and take actions (class imbalance), remove suspicious outliers 
%Visualize the data using scatterplots, histograms and box and whisker plots and look for extreme values
	\end{itemize}
	\subsection{Pre-analysis}
		%\item study relationships among features, what kind of algorithms are expected to perform well. 
		
	\subsection{Spot check algorithms}
		%random feature selection?
	%\subsection{
		
		%n some cases, we will need to run multiple models in parameters on the same training and test sets with the different priority of features, and compare the accuracy to choose an appropriate model for the given problem domain. These trials can run in parallel as there will not be any dependencies between these models. The complexity increases when we will have to tune the parameters of learning algorithms and evaluate across multiple executions to infer from the learning. The very fact that there is no dependency between the executions makes it highly parallelizable and requires no intercommunication. One of the examples of this use case is statistical significance testing. The usefulness of the parallel platforms is obvious for these tasks, as they can be easily performed concurrently without the need to parallelize actual learning and inference algorithms \cite{BOOK:6}

	\subsection{Mid-analysis}
	\subsection{Improve results}
		%exhaustive grid search? http://scikit-learn.org/stable/modules/grid_search.html#grid-search
	\subsection{Analysis}
		


		

\section{Tools}






\chapter{Data preparation process}
This chapter describes the process of preparing the dataset.

%\section{Data analysis} kanske ha denna sen
	%explain the process of analyzing the data

\section{Data cleaning} \label{sec:datacleaning}
	 In the dataset documentation provided by Trafikverket (see \ref{sec:provided_data}), there is a note on the measurements from station 1429 saying: "Unreasonable DST111 measurements from about 2016-11-15 to 2016-12-31". It was found in the measurements from station 1429 that the average difference between the DST111 and TIRS road surface temperature measurements were $~1.71 \celsius$ whereas in stations 1402 and 1431, which are the two geographically closest stations to 1429, the average differences were $0.614 \celsius$ and  $~0.54 \celsius$. In some cases, the difference between the measurements from the two temperature sensors in station 1429 were more than $40 \celsius$. This indicated that something may have been wrong with DST111, or some other instrument, at the time. The observations from 2016-11-15 to 2016-12-31 were removed from the workbook containing data from station 1429. This resulted in an average road surface temperature difference of $~0.92 \celsius$, which is still higher than the two nearby stations, but whether this was unreasonable or not could not be determined by the author. 

	In addition to removing suspicious outliers in the dataset, there are also cases where errors are explicitly reported by the sensors. As mentioned in the project delimitations (see \ref{sec:delimitations}), only non-error behavior is modelled in this project. As of such, every observation where at least one error is reported by any sensor, were removed. Figure \ref{img:histogram_surfstatus} shows that the DSC111 alone was malfunctioning for unknown reasons in more than $50 000$ observations. 

\begin{figure}[H] 
	\centering
	\includegraphics[width=0.8\textwidth]{media/HistogramSurfaceStatus.png}
	\caption{Histogram showing the distribution of the different surface condition types (see table \ref{table:discretevalues} to see what each code means).}
	\label{img:histogram_surfstatus}
\end{figure}

	More suspicious outliers may be present in the dataset that could potentially be identified by applying, for example, a confidence interval. But it was decided to not investigate this matter further since it was assumed that analytical knowledge in road condition data is needed to decide if an outlier represents an error or a correct abnormal value. After the data cleaning process, a total of 115180 observations remained, which means 56245 observations were removed.


\section{Feature selection}
	It was decided to investigate if timestamp should be excluded as a possible input feature, primarily because its values are higher than the rest of the features. Theory from \ref{sec:supervised_algorithms} suggests that some supervised learning algorithms, such as kNN, are sensitive to scaling. Timestamp is represented in the following format: mmddhhmm, which is interpreted as integers by any model that use it as an input feature. For example, an observation from station 1520 from 12/31/2016 23:30 have the following values: $\text{timestamp} = 12312330, \text{TIRS surface temperature}= 7.1, \dots \text{friction} = 0.74$. 

	Although the scaling of the timestamp feature is significantly different from the other input features, it seems to be correlated with other features. Figure \ref{img:correlations_noerr} shows how every feature is related to one another, it indicates that features such as road surface temperature and friction are linearly correlated with timestamp. The correlation may come as no surprise since the northern hemisphere is colder during winter-time, which means colder road surface temperatures and lower friction, and the other way around during warmer periods. 

\begin{figure}[H] 
	\centering
	\includegraphics[width=1\textwidth]{media/correlations_ver2.png}
	\caption{Depicts how the features are correlated to oneanother.}
	\label{img:correlations_noerr}
\end{figure}

	To test the relevance of using timestamp as input feature, an experiment was carried out to predict TIRS road surface temperature once using timestamp as input feature, and once without. The experiment runs the Cross-validation regression spot-checking setup (see \ref{sec:exp_setups}). Table \ref{table:timeinput} shows the result from the experiment.

	\begin{table}[H] %gör om med holdout
	\centering
	\caption{Experiment to see if timestamp is relevant to use as input feature. }
		\begin{tabular}[3]{l | c | c}
    			Algorithm & MSE not using time & MSE using time \\
    			\hline
			OLS & 1.23 & 1.22 \\
			CART & 1.21 & 1.84 \\
			kNN & 1.25 & 4.69 \\
			MLP & 1.06 & 55663.45 \\
			Lasso & 1.25 & 1.24 \\
			Random forest & 1.13 & 1.26 \\
			\hline
			Average total & 1.19 & 9279.13
			\label{table:timeinput}
		\end{tabular}
	\end{table}

	The results indicate that OLS and Lasso achieve slightly better performance by using time as input feature, whereas CART, kNN, Random forest and especially MLP, suffer in terms of performance when doing so. This is an indication that overall performance is improved by not using time as input feature, but the possibility of using timestamp as input feature was not ruled out yet. Section \ref{sec:transformation} deals with testing if timestamp can be scaled down to similar levels of other features and see if it improves overall performance or not.

\section{Data transformation} \label{sec:transformation}

	\subsection{Transforming timestamp}

	Table \ref{table:timeinput} shows significant improvement in overall performance in not using timestamp as input feature. However, figure \ref{img:correlations_noerr} shows that timestamp may be a relevant feature to include in that it shows correlation with other features. It was decided to test if a transformation of the timestamp feature could yield better performance than using it in its original form. The transformation involves using only month, and time of day from the timestamp feature, but to separate it into two columns so that they operate on lower numeric intervals. Figure \ref{img:transformation} shows the transformation. 

\begin{figure}[H] 
	\centering
	\includegraphics[width=0.8\textwidth]{media/transformation_time.png}
	\caption{Shows how timestamp as feature is transformed to two new features: month and hour.}
	\label{img:transformation}
\end{figure}

	The author assumes that month and time of day affect changes in weather conditions more than separate days within a month. Figure \ref{img:correlations_featureengi} shows the correlations among the features with month and hour used instead of timestamp. 

\begin{figure}[H] 
	\centering
	\includegraphics[width=1\textwidth]{media/correlations_featureengi_ver5.png}
	\caption{Correlations among the different features where month and hour are two new features that have replaced time. }
	\label{img:correlations_featureengi}
\end{figure}

	An experiment to predict TIRS road surface temperature with Cross-validation regression spot-checking setup was used to study the effects of the new transformation.

	\begin{table}[H] %gör om med holdout
	\centering
	\caption{Experiment to see the effects of transforming the time feature is relevant to use as input feature. }
		\resizebox{\textwidth}{!}{%
		\begin{tabular}[6]{l | c | c | c |c | c}
    			Algorithm & MSE using neither time nor new features & MSE using time & MSE using hour and month & MSE using month & MSE using hour \\
    			\hline
			OLS & 1.23 & 1.22 & 1.22 & 1.22 & 1.21 \\
			CART & 1.21 & 1.84 & 1.77 & 1.34 & 1.39 \\
			kNN & 1.25 & 4.69 & 1.08 & 1.24 & 1.15 \\
			MLP & 1.06 & 55663.45 & 1.30 & 1.06 & 1.08 \\
			Lasso & 1.25 & 1.24 & 1.23 & 1.24 & 1.23 \\
			Random forest & 1.13 & 1.26 & 1.01 & 1.02 & 1.12 \\
			\hline
			Total average & 1.19 & 9279.13 & 1.27 & 1.19 & 1.20
			\label{table:transformation_effect}
		\end{tabular}
		}
	\end{table}

	The results from table \ref{table:transformation_effect} indicate that performance is improved for all algorithms when the new features are used as a way to represent time rather than using timestamp. However, not using timestamp, hour or month, seem to have slightly higher performance than using both hour and month, but similar to using either of them. The slight differences in performance in not using any time-related features, as opposed to using the feature engineered ones, could be a a consequence of operating on different dimensionalities. For example MLP shows improved performance when fewer input features are used in table \ref{table:transformation_effect}. 

Although little to no overall performance improvement was made using the transformed time features, as opposed to not using any time-related features, it was decided to use the transformed input features as part of default input features for spot-checking experiments. This experiment is set up to test TIRS road temperature alone. The new features may prove more relevant with other target features and specific hyperparameter settings etc.


\subsection{Handling class imbalance} \label{sec:class_imbalance}
	From what was found in figure \ref{img:histogram_surfstatus}, road surface status suffers from class imbalance. But classifying road surface status is not a goal in this project. It is interesting to see if precipitation type have class imbalance as well, since classifying precipitation type is a goal in this project. Figure \ref{img:correlations_featureengi} shows the distribution of precipitation type in the dataset. 
	
	\begin{figure}[H] 
	\centering
	\includegraphics[width=1\textwidth]{media/histogram_prectype.png}
	\caption{Histogram showing the distribution of the different types of precipitation in the data (see table \ref{table:occurences_prectype} to see what each code means).}
	\label{img:correlations_featureengi}
	\end{figure}

	The exact number of occurences and a translation of what each code means is shown in table \ref{table:occurences_prectype}.

	\begin{table}[H]
	\centering
	\caption{Number of occurrences for different types of precipitation. }
		\begin{tabular}[3]{c | l | l}
    			Code & Precipitation type & N.o. occurrences \\
    			\hline
			1 & no precipitation & 94825 \\
			2 & rain with $>= 0 \celsius$ air temperature & 16094 \\
			3 & rain with $< 0 \celsius$ air temperature & 266 \\
			4 & snow & 3677 \\
			6 & rain and snow mixed & 316 
			\label{table:occurences_prectype}
		\end{tabular}
	\end{table}

	From what can be seen in table \ref{table:occurences_prectype}, precipitation type also have a significant class imbalance. The biggest difference is between no precipitation and rain and snow mixed, the former appears about 300 times more often than that the latter.

	An experiment was carried out to see if the class imbalance problem can be mitigated by using random oversampling or Smote as oversampling techniques (see information on oversampling techniques in \ref{sec:imbalancedtheory}). Smote runs on the default settings as seen in \cite{WEBSITE:23}. Both random oversampling and Smote were compared to a scenario where oversampling is not used. All three scenarios run the holdout classification spot-checking setup, to classify precipitation type. Holdout was used instead of $k$-fold in the three scenarios since it proved cumbersome to integrate oversampling with $k$-fold.

	\begin{table}[H]
	\centering
	\caption{Results from not using any oversampling techniques. }
		\begin{tabular}[5]{l | c | c | c | c}
    			Algorithm & Accuracy & $\overline{Precision}$ & $\overline{Recall}$ & $\overline{F1}$ \\
    			\hline
			LR & 0.81 & 0.24 & 0.21 & 0.21  \\
			kNN & 0.86 & 0.51 & 0.35 & 0.39  \\
			CART & 0.86 & 0.53 & 0.36 & 0.40 \\
			NB & 0.82 & 0.44 & 0.26 & 0.26  \\
			MLP & 0.86 & 0.47 & 0.33 &  0.38  \\
			Random forest & 0.86 & 0.56 & 0.37 & 0.41  \\
			\hline
			Total average & 0.85 & 0.46 & 0.31 & 0.34
			\label{table:no_oversampling}
		\end{tabular}
	\end{table}

	\begin{table}[H]
	\centering
	\caption{Results from using random oversampling as oversampling technique.}
		\begin{tabular}[5]{l | c | c | c | c}
    			Algorithm & Accuracy & $\overline{Precision}$ & $\overline{Recall}$ & $\overline{F1}$ \\
    			\hline
			LR & 0.48 & 0.33 & 0.55 & 0.31 \\
			kNN & 0.57 &  0.31 & 0.52 &  0.31 \\
			CART & 0.63 & 0.33 & 0.54 &  0.34 \\
			NB &  0.52 & 0.36 & 0.54 & 0.31 \\
			MLP & 0.62 & 0.36 & 0.60 & 0.36 \\
			Random forest & 0.63 & 0.33 & 0.54 &  0.34 \\
			\hline
			Total average & 0.58 & 0.34 & 0.55 & 0.33 
			\label{table:random_oversampling}
		\end{tabular}
	\end{table}

	\begin{table}[H]
	\centering
	\caption{Results from using Smote as oversampling technique.}
		\begin{tabular}[5]{l | c | c | c | c}
    			Algorithm & Accuracy & $\overline{Precision}$ & $\overline{Recall}$ & $\overline{F1}$ \\
    			\hline
			LR & 0.48 & 0.33 & 0.56 & 0.31 \\
			kNN & 0.68 &  0.31 & 0.47 &  0.34 \\
			CART & 0.72 & 0.33 & 0.48 &  0.35 \\
			NB &  0.53 & 0.36 & 0.53 & 0.31 \\
			MLP & 0.62 & 0.37 & 0.60 & 0.36 \\
			Random forest & 0.72 & 0.33 & 0.49 &  0.36 \\
			\hline
			Total average & 0.63 & 0.34 & 0.52 & 0.33 
			\label{table:smote_oversampling}
		\end{tabular}
	\end{table}
	
	Table \ref{table:no_oversampling} and \ref{table:random_oversampling} shows the effects of using random oversampling versus no oversampling: total average accuracy and precision is reduced, total average recall is higher, and total average $F1$ score is similar to the case when oversampling is not used. In the case of using Smote as oversampling technique as shown in table \ref{table:smote_oversampling}, it proved to have similar effects of using random oversampling. 

	Both precision and recall are important in this project when it comes to evaulating classification algorithm performance. Since no improvement was made on the collective score of precision and recall: $F1$, and overall accuracy was reduced when either of the oversampling techniques were tested, oversampling was ruled out as a possible solution to handle imbalanced classes. 

	Since the multiclass classification problem is but one of four subtasks in this project, additional effort was not put in to investigate if the imbalanced dataset problem can be solved in other ways.

\chapter{Results and analysis}
Describe the process of collecting data, training and implementing machine learning algorithms with different methods.


	
\section{Predicting road surface temperature (Track Ice road sensor)}
	\subsection{Input features correlation rankings}
	Table \ref{table:feature_comparison_tirs} shows a ranking of importance for each input feature based on a degree of correlation for each input feature to TIRS road surface temperature.

	\begin{table}[H]
		\centering
		\caption{Relevancy of each possible input feature to the target feature: TIRS road surface temperature. }
		\begin{tabular}[3]{c | l | l }
    			Relevancy ranking & Feature & Correlation score  \\
			 \hline
			1 & DST111 road surface temperature & 7688772.49 \\
			2 & road surface condition & 11048.87 \\
			3 & road friction & 6840.20 \\
			4 & month & 4300.00 \\
			5 & hour & 1968.02 \\
			6 & precipitation type & 1784.49 \\
			7 & precipitation amount & 238.91 \\
 
			\label{table:feature_comparison_tirs}
		\end{tabular}
	\end{table}

	The correlation ranking shows that the DST111 and DSC111 features are at the top.

	\subsection{Spot-checking}
		Next up is to find the optimal choice of features for each algorithm. A holdout regression spot-checking experiment was performed to see how each algorithm performs in terms of performance- and generalization score when the amount of features vary. 
	\begin{table}[H]
		\centering
		\caption{Results from spot-checking experiment on the top features for predicting precipitation amount on the test dataset. The results are shown as a tuple: ($MSE_{test}$, $MSE_{diff}$) where $MSE_{test}$ represents performance and $MSE_{diff} = (MSE_{test} - MSE_{train})$ shows the degree of overfitting, larger values of $MSE_{diff}$ indicate overfitting.}
		\resizebox{\textwidth}{!}{%
		\begin{tabular}[8]{l |c | c | c | c |c | c |c }
    			Algorithm & MSE top 7 & MSE top 6 & MSE top 5 & MSE top 4 & MSE top 3 & MSE top 2 & MSE top 1 \\
			 \hline 
			OLS 			& (1.19, 0.02) & (1.19, 0.02) & (1.19, 0.02)  & (1.20, 0.02)  & (1.22, 0.02) & (1.22, 0.02) & (1.22, 0.02) \\ \hline
			CART 		&  (1.36, 1.10) & (1.35, 1.08) & (1.31, 1.03) & (1.05, 0.30) & (1.03, 0.15) & (1.02, 0.09) & (1.01, 0.05) \\ \hline
			kNN 			& (0.94, 0.32) & (0.93, 0.32) & (0.92, 0.31) & (1.08, 0.18) & (1.16, 0.09) & (1.16, 0.06) & (1.21, 0.04) \\ \hline
			Backpropagation & (0.90, 0.01) & (0.86, 0.02) & (0.86, 0.01) & (0.97, 0.03) & (1.00, 0.02) & (1.02, 0.02) & (1.01, 0.01)\\ \hline
			Lasso 		& (1.24, 0.02) & (1.24, 0.02) & (1.24, 0.02) & (1.24, 0.02) & (1.24, 0.02) & (1.24, 0.02) & (1.24, 0.02) \\ \hline
			Random forest 	&  (1.01, 0.65) & (1.00, 0.66) & (1.01, 0.64) & (1.01, 0.23) & (1.01, 0.12) & (1.01, 0.08) & (1.01, 0.05)
 
			\label{table:spotcheck_tirs}
		\end{tabular}
		}
	\end{table}
	
	\begin{table}[H]
		\centering
		\caption{Shows an accumalated performance score $P_{acc}$ which is calculated from adding the contents of each tuple in \ref{table:spotcheck_tirs}:  $P_{acc} = MSE_{test} + MSE_{diff}$. Top results for each algorithm are highlighted and in case of ties the one using the fewest number of features is considered optimal.}
		\resizebox{\textwidth}{!}{%
		\begin{tabular}[8]{l |c | c | c | c |c | c |c }
    			Algorithm & $MSE_{diff}$ top 7 & $MSE_{diff}$ top 6 & $MSE_{diff}$ top 5 & $MSE_{diff}$ top 4 & $MSE_{diff}$ top 3 & $MSE_{diff}$ top 2 & $MSE_{diff}$ top 1 \\
			\hline
			OLS 			& 1.21 &  1.21 & \textbf{1.21} & 1.22 & 1.24 & 1.24 & 1,24 \\ \hline
			CART 		&  2.46 & 2.43 & 2.34 & 1.35 & 1.18 & 1.11 & \textbf{1.06} \\ \hline
			kNN 			& 1.26 & 1.25 & 1.25 & 1.26 & 1.25 & \textbf{1.22} & 1.25 \\ \hline
			Backpropagation &  0.91 & 0.88 & \textbf{0.87} & 1.00 & 1.02 & 1.04 & 1.02 \\ \hline
			Lasso 		& 1.26 & 1.26 & 1.26 & 1.26 & 1.26 & 1.26 & \textbf{1.26} \\ \hline
			Random forest 	&  1.66 & 1.66 & 1.65 & 1.24 & 1.13 & 1.09 & \textbf{1.06}
 
			\label{table:spotcheck_tirs_acc}
		\end{tabular}
		}
	\end{table}

	\subsection{Optimizing hyperparameters}
		Lasso, Backpropagation and kNN optimization experiments were run using their top performing input features. 

	\begin{table}[H]
		\centering
		\caption{Shows the results from optimizing hyperparameters of kNN, Backpropagation and Lasso}
		\resizebox{\textwidth}{!}{%
		\begin{tabular}[6]{l |c | c | c | c | c }
    			Algorithm & Default hyperparameter setting & Optimal setting & Default performance $P_{acc}$ & Optimal performance ($MSE_{test}$, $MSE_{diff}$) & Optimal performance $P_{acc}$ \\
			\hline
			kNN 			& $k = 5$ & $k = 64$ & 1.22 & (1.00, 0.04) & 1.04 \\ \hline
 			Backpropagation & n.o. hidden nodes: 100 & n.o. hidden nodes: 64 & 0.87 & (0.88, 0.01) & 0.89\\ \hline
			Lasso		& $\lambda = 1$ & $\lambda = 0.001$ & 1.26 & (1.22, 0.02) & 1.24
			\label{table:optimization_tirs}
		\end{tabular}
		}
	\end{table}

	As shown in \ref{table:optimization_tirs}, all results were improved when using new hyperparameter settings, except for Backpropagation which showed a slightly worse performance using optimized hyperparameters. However, since the default settings are used as well in the optimization process, it is believed that setting number of hidden nodes to 64 is indeed an optimal choice and thus an optimal overall performance was obtained.

	\subsection{Results and analysis}
		The best overall performances for each algorithm is shown in \ref{table:best_performances_tirs}.

	\begin{table}[H]
		\centering
		\caption{Shows the overall optimal settings and performances for each of the algorithms in predicting TIRS road surface temperature.}
		\resizebox{\textwidth}{!}{%
		\begin{tabular}[5]{l |c | c | c | c }
    			Algorithm & Optimal settings & Input features used & Best performance ($MSE_{test}$, $MSE_{diff}$) & Best performance $P_{acc}$ \\
			\hline
			OLS 				& default & top 5 & (1.19, 0.02) & 1.21 \\ \hline
			CART 			& default & top 1 & (1.01, 0.05) & 1.06\\ \hline
 			kNN 				& $k= 64$ & top 2 & (1.00, 0.04) & 1.04 \\ \hline
			Backpropagation		& n.o. hidden nodes: 64 & top 5 & (0.88, 0.01) & 0.89 \\ \hline
			Lasso			& $\lambda = 0.001$ & top 1 & (1.22, 0.02) & 1.24\\ \hline
			Random forest		& default & top 1 & (1.01, 0.05) & 1.06
			\label{table:best_performances_tirs}
		\end{tabular}
		}
	\end{table}

	Table \ref{table:best_performances_tirs} shows that Backpropagation has the lowest $P_{accc}$ score among all algorithms and is thus the the algorithm that performs best in terms of performance- and generalization when it comes to predicting TIRS road surface temperature. Followed by kNN, CART and Random forest.

\section{Classifying precipitation type (Optic Eye)}


\section{Predicting precipitation amount (Optic Eye)}
	\subsection{Input features correlation rankings}
	Table \ref{table:feature_comparison_tirs} shows a ranking of importance for each input feature based on a degree of correlation for each input feature to precipitation amount.

	\begin{table}[H]
		\centering
		\caption{Input features correlation rankings to the target feature: precipitation amount. }
		\begin{tabular}[3]{c | l | l }
    			Relevancy ranking & Feature & Correlation score  \\
			 \hline
			1 & road friction & 4478.75 \\
			2 & road surface condition & 2793.48 \\
			3 & DST111 road surface temperature & 234.64 \\
			4 & month & 60.36 \\
			5 & hour & 19.93 
			\label{table:feature_comparison_precamount}
		\end{tabular}
	\end{table}

		Table \ref{table:feature_comparison_precamount} shows that the DSC111 features have high correlation to precipitation amount while the time-related features have lower scores. 

	\subsection{Spot-checking}
		The next objective is to find an optimal choice of features for each algorithm. A holdout regression spot-checking experiment was performed to see how each algorithm performs in terms of performance- and generalization score when the amount of features vary. 

	\begin{table}[H]
		\centering
		\caption{Results from spot-checking experiment on the top features for predicting precipitation amount on the test dataset. The results are shown as a tuple: ($MSE_{test}$, $MSE_{diff}$) where $MSE_{test}$ represents performance and $MSE_{diff} = (MSE_{test} - MSE_{train})$ shows the degree of overfitting, larger values of $MSE_{diff}$ indicate overfitting.}
		\resizebox{\textwidth}{!}{%
		\begin{tabular}[6]{l |c | c | c | c |c }
    			Algorithm & top 5 features & top 4 features & top 3 features & top 2 features & top 1 features \\
			 \hline 
			OLS 			& (0.58, -0.06) & (0.58, -0.06) & (0.58, -0.06)& (0.58, -0.06) & (0.58, -0.06)\\ \hline
			CART 		& (1.01, 0.94) & (0.63, 0.18) & (0.98, 0.91) & (0.98, 0.80) & (0.62, 0.17)\\ \hline
			kNN 			& (0.60, 0.15) & (0.61, 0.07) & (0.58, 0.15) & (0.62, 0.15) & (0.63, 0.05)\\ \hline
			Backpropagation & (0.56, -0.06) & (0.57, -0.05) & (0.55, -0.05) & (0.56, -0.06) & (0.55, -0.06)\\ \hline
			Lasso 		& (0.61, -0.05) & (0.61, -0.05) & (0.61, -0.05) & (0.61, -0.05) & (0.61, -0.05) \\ \hline
			Random forest 	& (0.67, 0.52) & (0.59, 0.13) & (0.68, 0.51) & (0.71, 0.46) & (0.57, 0.11)
 
			\label{table:spotcheck_precamount_mse}
		\end{tabular}
		}
	\end{table}

	\begin{table}[H]
		\centering
		\caption{Shows an accumalated performance score $P_{acc}$ which is calculated from adding the contents of each tuple in \ref{table:spotcheck_precamount_mse}:  $P_{acc} = MSE_{test} + MSE_{diff}$. Top results for each algorithm are highlighted, in case of ties the one with fewest input features is considered optimal.}
		\resizebox{\textwidth}{!}{%
		\begin{tabular}[6]{l |c | c | c | c |c }
    			Algorithm & $P_{acc}$ top 5 features & $P_{acc}$ top 4 features & $P_{acc}$ top 3 features & $P_{acc}$ top 2 features & $P_{acc}$ top 1 features \\
			\hline
			OLS 			& 0.52 & 0.52 & 0.52 & 0.52 & \textbf{0.52} \\ \hline
			CART 		& 1.95 & 0.81 & 1.89 & 1.78 & \textbf{0.89} \\ \hline
			kNN 			& 0.75 & 0.68 & 0.73 & 0.77 & \textbf{0.68} \\ \hline
			Backpropagation & 0.50 & 0.52 & 0.50 & 0.50 & \textbf{0.49} \\ \hline
			Lasso 		& 0.56 & 0.56 & 0.56 & 0.56 & \textbf{0.56} \\ \hline
			Random forest 	& 1.19 & 0.72 & 1.19 & 1.17 & \textbf{0.68} 
 
			\label{table:spotcheck_precamount_acc}
		\end{tabular}
		}
	\end{table}

	\subsection{Optimizing hyperparameters}
		Lasso, Backpropagation and kNN optimization experiments were run using their top performing input features. 

	\begin{table}[H]
		\centering
		\caption{Shows the results from optimizing hyperparameters of kNN, Backpropagation and Lasso}
		\resizebox{\textwidth}{!}{%
		\begin{tabular}[6]{l |c | c | c | c  | c}
    			Algorithm & Default hyperparameter setting & Optimal setting & Default performance $P_{acc}$ & Optimal performance ($MSE_{test}$,$MSE_{diff}$) & Optimal performance $P_{acc}$ \\
			\hline
			kNN 			& $k = 5$ & $k = 64$ & 0.68 & (0.54, -0.05) & 0.49\\ \hline
 			Backpropagation & n.o. hidden nodes: 256 & n.o. hidden nodes: 64 & 0.49 & (0.55, -0.06) & 0.49 \\ \hline
			Lasso		& $\lambda = 1$ & $\lambda = 0.001$ & 0.56 & (0.58, -0.06) & 0.52
			\label{table:optimization_precamount}
		\end{tabular}
		}
	\end{table}

	Table \ref{table:optimization_tirs} shows that the results of kNN and Laso were improved using the optimal hyperparameter settings while Backpropagation produced the same result.

	\subsection{Results and analysis}
		The best overall performances for each algorithm is shown in \ref{table:best_performances_precamount}.

	\begin{table}[H]
		\centering
		\caption{Shows the overall optimal settings and performances for each of the algorithms in predicting precipitation amount.}
		\resizebox{\textwidth}{!}{%
		\begin{tabular}[5]{l |c | c | c | c }
    			Algorithm & Optimal settings & Input features used & Best performance ($MSE_{test}$, $MSE_{diff}$) & Best performance $P_{acc}$ \\
			\hline
			OLS 				& default & top 1 & (0.58, -0.06) & 0.52 \\ \hline
			CART 			& default & top 1 & (0.62, 0.17) & 0.79 \\ \hline
 			kNN 				& $k= 64$ & top 1 & (0.54, -0.05) & 0.49 \\ \hline
			Backpropagation		& n.o. hidden nodes: 256 & top 1 & (0.55, -0.06) & 0.49 \\ \hline
			Lasso			& $\lambda = 0.001$ & top 1 & (0.58, -0.06) & 0.52 \\ \hline
			Random forest		& default & top 1 & (0.57, 0.11) & 0,68
			\label{table:best_performances_precamount}
		\end{tabular}
		}
	\end{table}

	As is shown in \ref{table:best_performances_precamount}, Backpropagation and kNN are tied for the lowest $P_{acc}$ score among all algorithms. To break the tie, kNN is chosen before Backpropagation since it is generally less complex. This means that kNN is the algorithm that performs best in terms of performance- and generalization in predicting precipitation amount. 


\section{Predicting road surface temperature (DST111)}
	\subsection{Input features correlation rankings}
	Table \ref{table:feature_comparison_dst111} shows a ranking of importance for each input feature based on a degree of correlation for each input feature to DST111 road surface temperature.

	\begin{table}[H]
		\centering
		\caption{Relevancy of each possible input features to the target feature: DST111 road surface temperature. }
		\begin{tabular}[3]{c | l | l }
    			Relevancy ranking & Feature & Correlation score  \\
			 \hline
			1 & road surface condition & 11636.00 \\
			2 & road friction & 7411.80 \\
			3 & month & 4990.95 \\
			4 & precipitation type & 1897.37 \\
			5 & hour & 1784.49 \\
			6 & precipitation amount & 234,64 \\
 
			\label{table:feature_comparison_dst111}
		\end{tabular}
	\end{table}

	The correlation ranking shows that the DST111 and DSC111 features are at the top.

	\subsection{Spot-checking}
		Next up is to find the optimal choice of features for each algorithm. A holdout regression spot-checking experiment was performed to see how each algorithm performs in terms of performance- and generalization score when the amount of features vary. 
	\begin{table}[H]
		\centering
		\caption{Results from spot-checking experiment on the top features for predicting DST111 road surface temperature on the test dataset. The results are shown as a tuple: ($MSE_{test}$, $MSE_{diff}$) where $MSE_{test}$ represents performance $MSE_{diff} = (MSE_{test} - MSE_{train})$ shows the degree of overfitting, larger values of $MSE_{diff}$ indicate overfitting.}
		\resizebox{\textwidth}{!}{%
		\begin{tabular}[7]{l |c | c | c | c |c | c  }
    			Algorithm & top 6 features & top 5 features & top 4 features & top 3 features & top 2 features & top 1 features \\
			 \hline 
			OLS 			& (70.45, 0.20) & (70.50, 0.22) & (71.69, 0.21) & (71.70, 0.21) & (75.25, 0.06) & (75.64, 0.12) \\ \hline
			CART 		& (10.47, 1.54) & (10.27, 1.03) & (20.06, 0.50) & (20.56, 0.48) & (60.20, 0.01) & (60.93, -0.12) \\ \hline
			kNN 			& (11.83, 0.65) & (11.90, 0.82) & (22.60, 0.44) & (23.36, 0.34) & (62.68, -0.22) & (61.66, -0.13) \\ \hline
			Backpropagation & (11.46, 0.27) & (13.14, 0.13) & (22.33, 0.34) & (22.44, 0.30) & (61.00, -0.15) & (62.49, -0.20) \\ \hline
			Lasso 		& (71.43, 0.19) & (71.43, 0.19) & (72.65, 0.20) & (72.65, 0.20) & (76.38, 0.04) & (76.38, 0.04) \\ \hline
			Random forest 	& (10.16, 1.11) & (10.16, 0.86) & (20.04, 0.46) & (20.56, 0.46) & (60.20, 0.00) & (60.93, -0.12) 
 
			\label{table:spotcheck_dst111}
		\end{tabular}
		}
	\end{table}
	
	\begin{table}[H]
		\centering
		\caption{Shows an accumalated performance score $P_{acc}$ which is calculated from adding the contents of each tuple in \ref{table:spotcheck_dst111}:  $P_{acc} = MSE_{test} + MSE_{diff}$. Top results for each algorithm are highlighted.}
		\resizebox{\textwidth}{!}{%
		\begin{tabular}[7]{l |c | c | c | c |c | c  }
    			Algorithm &  $P_{acc}$ top 6 features& $P_{acc}$ top 5 features & $P_{acc}$ top 4 features & $P_{acc}$ top 3 features & $P_{acc}$ top 2 features & $P_{acc}$ top 1 features \\
			 \hline 
			OLS 			& \textbf{70.65} & 70.72 & 71.90 & 71.91 & 75.31 & 75.52 \\ \hline
			CART 		& 12.01 & \textbf{11.30} & 20.56 & 21.04 & 60.21 & 60.81 \\ \hline
			kNN 			& \textbf{12.48} & 12.72 & 23.04 & 23.70 & 62.46 & 61.53 \\ \hline
			Backpropagation & \textbf{11.73} & 13.27 & 22.67 & 22.74 & 60.85 & 62.29 \\ \hline
			Lasso 		& \textbf{71.62} & 71.63 & 72.85 & 72.85 & 76.42 & 76.42 \\ \hline
			Random forest 	& 11.27 & \textbf{11.02} & 20.50 & 21.02 & 60.20 & 60.81 
 
			\label{table:spotcheck_dst111_acc}
		\end{tabular}
		}
	\end{table}

	\subsection{Optimizing hyperparameters}
		Lasso, Backpropagation and kNN optimization experiments were run using their top performing input features. 

	\begin{table}[H]
		\centering
		\caption{Shows the results from optimizing hyperparameters of kNN, Backpropagation and Lasso}
		\resizebox{\textwidth}{!}{%
		\begin{tabular}[6]{l |c | c | c | c | c }
    			Algorithm & Default hyperparameter setting & Optimal setting & Default performance $P_{acc}$ & Optimal performance ($MSE_{test}$,$MSE_{diff}$) & Optimal performance $P_{acc}$ \\
			\hline
			kNN 			& $k = 5$ & $k = 32$ & 12.48 & (10.69, 0.43) & 11.12 \\ \hline
 			Backpropagation & n.o. hidden nodes: 256 & n.o. hidden nodes: 64 & 11.73 & (11.19, 0.29) & 11.48 \\ \hline
			Lasso		& $\lambda = 1$ & $\lambda = 0.001$ & 71.62 & (70.46, 0.21) & 70.67
			\label{table:optimization_dst111}
		\end{tabular}
		}
	\end{table}

	\subsection{Results and analysis}
	Table \ref{table:best_performances_dst111} shows the best performances of each algorithm in predicting DST111 road surface temperature. 


	\begin{table}[H]
		\centering
		\caption{Shows the overall optimal settings and performances for each of the algorithms in predicting precipitation amount.}
		\resizebox{\textwidth}{!}{%
		\begin{tabular}[5]{l |c | c | c  | c}
    			Algorithm & Optimal settings & Input features used & Best performance ($MSE_{test}$, $MSE_{diff}$) & Best performance $P_{acc}$ \\
			\hline
			OLS 				& default & top 6 & (70.45, 0.20) & 70.65 \\ \hline
			CART 			& default & top 5 & (10.27, 1.03) & 11.30 \\ \hline
 			kNN 				& $k= 32$ & top 6 & (11.83, 0.65) & 12.48 \\ \hline
			Backpropagation		& n.o. hidden nodes: 256 & top 6 & (11.46, 0.27) & 11.73 \\ \hline
			Lasso			& $\lambda = 0.001$ & top 6 & (71.43, 0.19) & 71.62 \\ \hline
			Random forest		& default & top 5 & (10.16, 0.86) & 11.02
			\label{table:best_performances_dst111}
		\end{tabular}
		}
	\end{table}

		From \ref{table:best_performances_dst111} it is revealed that Random forest has the lowest $P_{acc}$ score and is thus the best algorithm in predicting DST111 road surface temperature.




\chapter{Summary}
\emph{This chapter aims at summarizing the objective of this project as well as the results obtained for the given objective.}

\section{Recap and solutions to project subtasks}
	This project has dealt with investigating whether or not road weather station sensors can be modelled using supervised machine learning algorithms. The sensors that are of interest to model are as follows:
	\begin{itemize}
		\item{Optic Eye:} Measures precipitation type and precipitation amount.
		\item{Track Ice Road Sensor:} Dug into the road. Measures road surface temperature.
		\item{DST111:} Measures road surface temperature via infrared.
	\end{itemize}
	
	The idea was to try to model each sensor from data provided by the others, except for Track Ice Road Sensor since this may be removed in the future. In addition to the sensor above, any model may be trained using data from the following sensors as well:
	\begin{itemize}
		\item{DSC111:} Measures road surface condition and road friction.
		\item{MS4:} Timestamp (this was made into two input features: hour and month in \ref{sec:transformation}).
	\end{itemize}

	The objective of the project as seen in \ref{sec:objective} is as follows:

	"The objective is to find optimal supervised learning models, in terms of performance and generalization, which models the behavior of the following sensors: Optic Eye, Track Ice Road Sensor and DST111 where each sensor is trained from observations made from other the other sensors but its own. In addition to the forementioned sensors, except for Track Ice Road Sensor, any model may train on the following input features as well: measurement timestamp, road friction and road surface condition."

	The objective was broken down into four subtasks. Each subtask and their solutions are covered in the following subsections.

	\subsection{Classifying precipitation type}
		The first subtask mentioned in \ref{sec:objective} is:
	
	Find the algorithm among supervised learning algorithms, that can be used to classify \textbf{precipitation type}, whose performance and generalization score is best. The data may contain the following input features: 
			\begin{itemize}
				\item measurement timestamp
				\item DST111 road surface temperature
				\item road friction
				\item road surface condition
			\end{itemize}

	\subsection{Predicting precipitation amount}
		The second subtask mentioned in \ref{sec:objective} is:
	
	Find the algorithm among supervised learning algorithms, that can be used to predict \textbf{precipitation amount}, whose performance and generalization score is best. The data may contain the following input features: 
			\begin{itemize}
				\item measurement timestamp
				\item DST111 road surface temperature
				\item road friction
				\item road surface condition
			\end{itemize}

	The results from \ref{sec:results_precamount} shows that the best algorithm in terms of performance and generalization in predicting precipitation amount is kNN which obtained a performance score of $MSE_{test} = 0.54$.

	\subsection{Predicting Track Ice Road Sensor road surface temperature}
		The third subtask mentioned in \ref{sec:objective} is:
	
	Find the algorithm among supervised learning algorithms, that can be used to predict \textbf{Track Ice Road Sensor road surface temperature}, whose performance and generalization score is best. The data may contain the following input features: 
			\begin{itemize}
				\item measurement timestamp
				\item precipitation type
				\item precipitation amount
				\item DST111 road surface temperature
				\item road friction
				\item road surface condition
			\end{itemize}

	It was found in \ref{sec:results_tirs} that the best algorithm in terms of performance and generalization for predicting TIRS road surface temperature is Backpropagation which obtained a performance score of $MSE_{test} = 0.88$.

	\subsection{Predicting DST111 road surface temperature}
		The fourth subtask mentioned in \ref{sec:objective} is:
	
	Find the algorithm among supervised learning algorithms, that can be used to predict \textbf{DST111 road surface temperature}, whose performance and generalization score is best. The data may contain the following input features: 
			\begin{itemize}
				\item measurement timestamp
				\item precipitation type
				\item precipitation amount
				\item road friction
				\item road surface condition
			\end{itemize}

	From the results in \ref{sec:results_dst111} it was found that Random forest was the top performing algorithm in terms of performance and generalization. It obtained a performance score of $MSE_{test} = 10.16$.
	


\chapter{Conclusions and recommendations}

\section{Conclusions}

\section{Recommendations}


%\chapter{Discussion}

\section{Thesis process}

\section{Validity and reliability}
Validity and reliability of the conclusions. Needed?
\section{Future work}


\printbibliography
\end{document}
